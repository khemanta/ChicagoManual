\chapter*{Preface} \lettrine[lines=2, lhang=0.33, loversize=0.25]{T}{his
dissertation contains} selected results from my doctoral studies.  The
main focus of these studies has been on applying \ac{RICS} for
determining properties of diffusion in cardiomyocytes. In the beginning
stages of this project an attempt was made to use existing equipment
available at Tallinn University of Technology. Unfortunately, after
months of trials it became clear that this equipment was not able to
carry out the experimental protocol we had developed in a satisfactory
manner.  As a result of this, we designed and built a confocal
microscopy setup of our own, optimized for performing \ac{RICS}
experiments. 

Before the building and testing was complete, however, I travelled to
New York in order to spend 8 months at Mount Sinai School of Medicine in
the laboratory of Dr. Eric A. Sobie. This side-project, although also
involving confocal microscopy and cardiac muscle cells, was quite
different from what I had previously been involved in. The collaboration
developed into an article on hypertrophic cardiomyocytes (\PaperII) and
resulted in the release of a software package aimed at calcium transient
analysis.

On my return to Estonia the work on \ac{RICS} resumed. Although the
technical difficulties with the equipment had now been resolved,
optimization of the experimental protocol remained. After a year of
fine-tuning and improvements, experiments started to become
reproducible. Parallel to conducting experiments, I started working on a
stochastic computational model of diffusion that I employed for running
\ac{RICS} simulations \insilico in order to obtain further insights into
the results yielded by biological experiments. The results of this
combined approach to study diffusion in cardiomyocytes are presented in
\PaperIII and form the main body of this text. 

As \ac{RICS} is only one method used in our laboratory in its quest to
unravel the subtleties of cardiac energetics, it is only suitable that
we would monitor other developments in the field. One direct result of
this is an article (\PaperI) reviewing recent results in postnatal
development of energetic microdomains in mouse. 
%During the foray into the subtleties of \ac{RICS} we stumbled upon
%several interesting aspects and applications of it that have not yet
%had an opportunity to be compiled into a manuscript. 
%\pagebreak
\newpage
%\vspace{-.3cm}
%\subsection*{Acknowledgements}

\addcontentsline{toc}{chapter}{\tocEntry{Acknowledgements}}
\chapter*{Acknowledgements} \lettrine[lines=2, lhang=0.33,
loversize=0.25]{M}{y curiosity towards} the world of science got a major boost when my
father started purchasing the Scientific American journal in the early
nineties so that I could satisfy my
appetite for reading about intriguing new aspects of this world. The books that accumulated in my home on topics such as cellular automata, chaos,
complexity and fractals captured my attention and kindled my interest in
these fascinating fields. I am grateful for his effort in immersing me in
these topics and his interest in my ongoing scientific endeavours. The
support of my mother for my studies and interests was equally crucial in
ensuring that I reach where I am today. I am thankful for the help and
guidance from my parents in everything I have undertaken.

During my bachelor studies, after completing a course given by Professor Jüri
Engel\-brecht, I had the honour of being asked to join the
Institute of Cybernetics. Having read his books on chaos and complexity
it was a thrill to take one of his classes. It was beyond my wildest
dreams, however, to be invited to work on these very same topics I had
been reading about and tinkering with on my own just a few years prior.
I am thankful to Professor Engelbrecht also for acquainting me with my
supervisor Dr. Marko Vendelin, whose enthusiasm and energy accompanied
me on my journey into cardiac energetics. Marko's advice and knowledge have
made me feel confident when navigating the stormy and unpredictable seas
of biophysical research. Also, I will be forever in his debt for
introducing me to the most special person in my life.

It has been a pleasure working together with all the members of the
Systems Biology lab: the original PhD student gang David, Mari and Mervi
with whom we have been through thick and thin, and the latecomers Martin,
Niina, Na\-talja, Jelena, Merle and Päivo. This work
would not have been possible without their help. My gratitude also belongs to Dr. Rikke
Birkedal for her efforts in ensuring that I have cells to perform
experiments on (especially during the do-or-die week during the
christmas break in 2011) and to Dr. Pearu Peterson for keeping a
watchful eye that the temptation of fuzzy language is fought and that
only words based on facts ever get put down on paper. 

I am grateful to Dr. Eric A. Sobie from the Mount Sinai School of
Medicine for offering me the opportunity to work with him for 8
months in New York and for stepping up to help in a bleak
situation caused by unfortunate circumstances. 

Lastly, without the unconditional support and love from my wife Hena, the
last years of this journey would have been an immeasurably more arduous
undertaking.  

Financial support from the Wellcome Trust, the Estonian Science Foundation, as
well as the Archimedes Foundation is appreciated.
