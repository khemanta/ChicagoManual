\chapter*{Summary}
\lettrine[lines=2, lhang=0.33, loversize=0.25]{T}{his
dissertation explores} the fascinating world of cardiac function
and dysfunction. Through application of various experimental and
computational techniques new insights into aspects of intracellular diffusion
restrictions, causes of hypertrophy, and development of cardiac
energetics are gained. Experiments on rat cardiomyocytes using \acl{RICS} reveal a
counterintuitive result, whereby diffusion of a smaller fluorescent
molecule is restricted more than that of a larger one, when
comparing diffusion in the cytosol to that in solution. A stochastic
computational model of diffusion is applied to find a
possible explanation for this result. Modeling results suggest the existence
of regularly placed semi-permeable barriers in the cardiomyocyte
situated $\sim$ 1 $\mu$m apart and having very low permeabilities.
Such structures in the intracellular environment could enact a
significant role in the function of cardiomyocytes, as well as in
dysfunctional states of the heart. Also discussed are experiments 
probing the causes of hypertrophic cardiomyopathy with the aim of
studying mutations that affect signalling cascades and cause
abnormalities in cardiomyocyte calcium handling.

