\chapter*{Kokkuv\~ote}
\lettrine[lines=2, lhang=0.33, loversize=0.25]{K}{\"aesolevas v\"aitekirjas
sisenetakse} südame funktsioneerimise kee\-ru\-lis\-se aga köit\-vasse
maailma. Raken\-dades eri\-ne\-vaid eks\-peri\-men\-taal\-seid ja
arvu\-tus\-likke meeto\-deid saa\-dakse uusi teadmisi rakusiseste
diffusioonitakistuste, südame hüpertroofia ja südameenergeetika
arenguliste muutuste valdkondadest. Raster
korre\-lat\-sioon-spektroskoopia katsed roti südamelihasrakus näitavad,
et väiksemate molekulide difusioon on raku sees rohkem takistatud kui
suuremate molekulide difusioon. Sellele intuitisioonivastasele
tulemusele  leitakse seletus kasutades stoh\-hastilist arvutuslikku
mudelit modelleerimaks rakusisest takistatud diffusiooni.
Modelleerimistulemused vihjavad korrapäraste, osaliselt läbitavate
barjääride ole\-ma\-so\-lu\-le südamelihasrakkudes. Nimetatud barjäärid
asuksid teine-teisest $\sim$ 1 $\mu$m kaugusel ja oleks suhteliselt
raskesti lä\-bi\-tavad. Sellised rakusisesed struktuurid või\-vad omada
olulist rolli nii süda\-me\-li\-has\-rakkude normaalses toimimises kui
ka patoloogilistes juhtudes. Antud töö esitab veel katse\-tulemusi
südame hüper\-troofia põh\-juste uuringutest. Saadud tulemused
selgitavad kuidas hüper\-troofiat põh\-justavad mutat\-sioonid
mõ\-ju\-ta\-vad signaali\-kaskaade ning põh\-jus\-tavad ohtlikke
kõrva\-le\-kal\-deid kaltsium\-iringluse nor\-maal\-sest toi\-mi\-mi\-sest
sü\-da\-me\-lihas\-rakku\-des.


\[\]

