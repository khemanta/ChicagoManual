
\chapter{Conclusions}
\label{partCon:conclusions}
\setlength{\parskip}{0.8ex plus 0.5ex minus 0.1ex}

\lettrine[lines=2, lhang=0.33, loversize=0.25]{T}{his dissertation
explores various} aspects of cardiac function and dysfunction. The main conclusions deriving from this work are listed below, grouped
according to the research topic.

\vspace{-0.2cm}





\section*{Extended RICS technique}
\phantomsection\addcontentsline{toc}{section}{Extended RICS technique}
%to save the list item count
\newcounter{saveenum}
\begin{enumerate}[a)]
    \item The extended \ac{RICS} protocol developed during these doctoral studies
is a useful tool for determining concentrations, diffusion coefficients
and diffusional anisotropy of fluorescent dyes in cells.

\item The \ac{RICS} method, when applied to estimating diffusion in
    water, is
able to reproduce the known \acp{DC} of the dye ATTO655-COOH with high
accuracy.

\item The ``poking'' technique is a good alternative to saponin
permeabilization. It requires dedicated hardware in form of
controllable micromanipulators and extra preparation in order to ensure cells
remain on the coverslip after mechanical permeabilization. However in
cases like anisotropic \ac{RICS} measurements, where one protocol run can
take several hours, good viability of cells is a necessity.

\item Taking into account the possibility of fluorescent molecules entering
into a triplet state enhanced the ability of the method to estimate
\acp{DC}.

\item Assuming two sub-species of the same dye to be diffusing in the cytosol,
one slowly diffusing bound form and one freely diffusing form improved
the results greatly when estimating \acp{DC} of \ATP. In case of \DEX\
adding a second component was not necessary and did not improve fits to
\ac{CF} curves. This indicates that \ATP\ is more actively binding to
intracellular proteins or structures, whereas \DEX\ remains relatively
inert.
\setcounter{saveenum}{\value{enumi}}
\end{enumerate}




\section*{Estimation of diffusion coefficients of fluorescent dyes}
\phantomsection\addcontentsline{toc}{section}{Estimation of diffusion coefficients of fluorescent dyes}

\begin{enumerate}[a)]
    \setcounter{enumi}{\value{saveenum}}
\item Employing the \ac{RICS} method, \acp{DC} of \ATP\ were determined to be 326$\pm$13, 195$\pm$8, 24$\pm$6,
35$\pm$8 $\mu$m$^2$/s in water, artificial intracellular solution,
transverse direction and longitudinal direction in the cardiomyocyte,
respectively. Similarly, \acp{DC} of \DEX\ were determined to be 62$\pm$1, 53$\pm$1,
16$\pm$2, 19$\pm$3 $\mu$m$^2$/s in water, artificial intracellular solution,
transverse direction and longitudinal direction in the cardiomyocyte,
respectively. 

\item The relative decline in \acp{DC} when transitioning from solution to the
intracellular environment was higher for \ATP\ than for \DEX.
Considering the fact that the molecular mass of \ATP\ is $\sim$9 times less than
that of \DEX\ makes this result counterintuitive, whereby a
smaller molecule is hindered more in the cytosol than the larger one.
\setcounter{saveenum}{\value{enumi}}
\end{enumerate}



\section*{Mathematical modelling of restricted diffusion}
\phantomsection\addcontentsline{toc}{section}{Mathematical modelling of restricted diffusion}
\begin{enumerate}[a)]
    \setcounter{enumi}{\value{saveenum}}
    \item The stochastic model of restricted diffusion with regular permeable
barriers is able to reproduce the experimentally found \acp{DC} for
\ATP\ and \DEX\ when: barriers in the model are placed relatively close
$<$1 $\mu$m apart, permeability of these barriers is low (containing
$\sim$ 1$\ldots$40 pores of radius 7$\ldots$30 nm per $\mu$m$^2$),
diffusion coefficient in the inter barrier space is lower than diffusion
in solution by a factor 0.8$\ldots$1.

\item The model geometry used in this work is not the only one that is able to
reproduce experimentally obtained \acp{DC}. Our choice was motivated by
the relative simplicity of the employed geometry of the model. 

\item Computational experiments applying \ac{RICS} to 1D diffusion reveal
that barriers affect estimation of \acp{DC} even if the region used for
estimation does not contain a barrier itself but is close to one. The
distance at which this effect is observable depends on the permeability
of the barrier, the \ac{DC} of the dye in the inter barrier space
and the size of the segment used for apparent \ac{DC} estimation.

\setcounter{saveenum}{\value{enumi}}

\end{enumerate}
\section*{Cardiomyocyte hypertrophy in Noonan syndrome}
\phantomsection\addcontentsline{toc}{section}{Cardiomyocyte hypertrophy in Noonan syndrome}

\begin{enumerate}[a)]
    \setcounter{enumi}{\value{saveenum}}
    \item The study conducted in \PaperII established a link between
        \acf{HCM} and overexpression or mutation of \acl{RAF1}. Both
        wild-type  and L613V \ac{RAF1} signal through calcineurin to
        induce hypertrophy in affected cardiomyocytes.
    \item RAF1 impairs calcium signalling by down-regulating \ac{SERCA}
        but does not affect expression levels of other proteins involved
        in intracellular calcium signalling.

    \item Wild-type \ac{RAF1} overexpression and L613V both result in
        arrhythmogenic dysregulation of calcium signalling, indicated by spontaneous calcium 
transients during electrical stimulation, slower decay rates and, in case of L613V, an increased sarcoplasmic reticulum calcium load.
\item Treatment with cyclosporine-A  prevented both wild-type \ac{RAF1} and L613V induced \ac{HCM}.

  \setcounter{saveenum}{\value{enumi}}

\end{enumerate}

\section*{Development of energetic microdomains}
\phantomsection\addcontentsline{toc}{section}{Development of energetic microdomains}
\begin{enumerate}[a)]
    \setcounter{enumi}{\value{saveenum}}
    \item Regeneration of energy metabolites through glycolysis is an
        important factor, especially during maturation, and should be
        included in the analysis of developmental changes in energy
        transfer pathways.
    \item Computational models could help interpret existing data and
        could reveal novel aspects of the interplay between various
        factors during cell maturation. Using statistical methods,
        different mathematical models could be compared to establish the
        existence of metabolite pools or compartmentation in the
        \ac{CM}.
\end{enumerate}



\endinput
