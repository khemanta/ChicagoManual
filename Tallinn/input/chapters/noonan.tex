\chapter{Hypertrofic cardiomyopathy in RAF1 mutants}
\label{ch:noonan}
\lettrine[lines=2, lhang=0.33, loversize=0.25]{N}{oonan and LEOPARD syndromes} are developmental disorders characterized
by distinct facial features, chest deformities, short stature and a wide array
of congenital heart diseases \cite{Allanson_87_JMedGenet_24_p9}. These diseases are linked to a
variety of germline gain-of-function mutations. Mutations in \ac{RAF1} have
been determined to be the cause behind 3-5\% of cases of Noonan and LEOPARD 
syndrome in affected individuals \cite{P_07_NatGenet_39_p1007}. 90-95 \% of patients with Noonan
\ac{RAF1} mutations exhibit \ac{HCM}, whereas Noonan
syndrome caused by mutations in other locations often lead to different
cardiac defects. This suggests an important role for \ac{RAF1} in modulation
of cardiac hypertrophy. 
\ac{RAF1} is a serine/theonine kinase which is part of a signalling pathway
involved in transducing signals from the cell membrane to the nucleus.
\ac{RAF1} regulation is intricate and is affected by protein-protein interactions,
localization of the protein and phosphorylation at multiple residues. In \PaperII the
ability of \ac{RAF1} to regulate hypertrophy in cardiomyocytes was studied. 

Three common \ac{RAF1} mutations were explored in order to determine whether
and how these ultimately result in hypertrophy. It was established that two of
the studied mutations, S257L and L613V, localized close to
phosphorylation sites S259 and S621, promote \ac{HCM}. The third mutation
(D486N) was found not to cause \ac{HCM}. Overexpression of wild-type
\ac{RAF1} also stimulated hypertrophy. Treatment with cyclosporine-A
(inhibitor of calcineurin) prevented both wild-type \ac{RAF1} and L613V
induced \ac{HCM}. 

Wild-type \ac{RAF1} overexpression and L613V both result in dysregulation of
\ce{Ca^2+} signalling which was exemplified by spontaneous \ce{Ca^2+}
transients, slower decay rates and, in case of L613V, an increased sarcoplasmic reticulum
\ce{Ca^2+} load. This finding was attributed to down-regulation of
\ac{SERCA} through overexpression of wild-type and L613V \ac{RAF1}.
%Considering the vital role of \ac{SERCA} in intracellular calcium the
%dysfunction caused by its down-regulation is not surprising. 

The study conducted in \PaperII established a link between \ac{HCM} and
arrhythmogenic \ce{Ca^2+} dysregulation, found an interaction between
calcineurin and \ac{RAF1}, and identified a link for mutant \ac{RAF1} induced
pathological \ac{HCM}.


