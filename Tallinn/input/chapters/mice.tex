\chapter{Development of cardiac energetics in postnatal mouse heart}
\label{ch:mice}
\lettrine[lines=2, lhang=0.33, loversize=0.25]{C}{ompartmentation of
energy metabolites} in cardiomyocytes aims to
lessen the impact of abrupt changes in energy consumption and allows for
high phosphorylation potential to exist where it is most required. Not
much is known about the nature, structure and development of barriers responsible for such
compartmentation. In the study reviewed in \PaperI we summarize and
comment on some of the aspects presented in a extensive investigation
into how the structural and energetic properties of mouse heart muscle
change during postnatal development \cite{Piquereau_10_JPhysiol_Lond__588_p2443}.

The original paper concluded that energetic microdomains are formed very
early in postnatal development and that the maturation of cellular
architecture has an important role to play ensuring maximal flexibility
in regulation of \ac{ATP} production by mitochondria. 
%The maximal
%throughput of energy transfer between mitochondria and myosin ATPases
%was found to correlate with changes in cytoarchitecture. \ac{CK}
%supported energy transfer, in contrast, was found to depend on
%localization and expression of CK. 
%
Although an impressive amount of work had been invested by the authors of
the original paper, we found several issues with the conclusions and questions left
unanswered by the authors. We pointed out the misinterpretation of one
derived variable used by the authors to propose an increase in the
functional coupling between the \ac{ANT} and mitochondrial \ac{CK}. We
proposed an alternative explanation of an increase in diffusion
restrictions in the cytosol and suggested an experiment for measuring
the coupling between \ac{ANT} and mitochondrial \ac{CK}. Furthermore, we
used a different approach from the original authors to interpret their
data and to focus on the changes of the role of energy supply pathways
by eliminating auxiliary effects. We also suggested some possible directions
for future research in this area. Namely, accounting for the changing role of glycolysis
during maturation and employing computational models in analysing the
data, which could help unravel the interplay between different factors
during cell maturation.
