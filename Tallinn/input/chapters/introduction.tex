\chapter{Introduction}
\lettrine[lines=2, lhang=0.33, loversize=0.25]{T}{he subjects that have
been} explored during my doctoral studies, 
employ various methods and approaches. They all, however, deal 
with the common underlying topic of function and dysfunction of cardiac
cells. 

The first of these subjects relates to the study of how intracellular
compartmentation and diffusion restrictions influence cardiac
energetics. The existence of diffusion restrictions has been shown in
numerous experiments. For example, studies on caffeine induced tension
transients have revealed the restricted access of externally supplied
\ac{ATP} to intracellular enzymes consuming \ac{ATP} \cite{Kaasik_01_CircRes_89_p153}.
Measurements of \ce{O_2} consumption by mitochondria during oxidative
phosphorylation in permeabilized cardiomyocytes also
indicate existence of diffusion restrictions between mitochondria and
extracellular solution
\cite{Sepp_10_BiophysJ_98_p2785,Kummel_88_CardiovascRes_22_p359}. The presence of such compartmentation can
be visualized by looking at oscillations of mitochondrial
membrane potential induced by \ac{ROS}
\cite{Kurz_10_ProcNatlAcadSci_107_p14315} and modulation of the \ac{ATP}
sensitive potassium channel by phosphocreatine
\citep{Abraham_02_JBiolChem_277_p24427}. Compartmentation caused by such restrictions
and how this affects energy transfer in the heart is a clinically
relevant problem \cite{Gudbjarnason_70_JMolCellCardiol_1_p325} and
concentrations of molecules involved in
intracellular energy transfer can be used as a predictive indicator of
mortality in dilated cardiomyopathy patients
\cite{Neubauer_07_NEnglJMed_356_p1140}.

Substantial diffusion restrictions have so far been suggested from
indirect measurements as mentioned above. In skeletal muscle,
\ac{NMR} experiments have found diffusion of \ac{ATP} to be
hindered compared to diffusion in water
\cite{Kushmerick_69_Science_80-__166_p1297,deGraaf_00_BiophysJ_78_p1657}.
However, the decrease in diffusion coefficients found in these \ac{NMR}
experiments is not big enough to explain the experimental results from
oxygraphy measurements listed above. According to computational
modelling attempting to reproduce results from oxygen consumption
experiments
\cite{Saks_03_BiophysJ_84_p3436}, a decrease several times higher than
that found in \ac{NMR} experiments in the diffusion coefficients of
adenine nucleotides is necessary to account for the results from
oxygraphy measurements.


Employing a novel \ac{FCS} based technique termed \acf{RICS}, diffusion
of fluorescent \ac{ATP} 
has been measured in cardiac myocytes \cite{Vendelin_08_AmJPhysiolCellPhysiol_295_pC1302}. The \ac{RICS} method makes use of
the fact that in a confocal image each pixel is separated from another
not only in distance but also in time
\cite{Digman_11_AnnuRevPhysChem_62_p645,Digman_05_BiophysJ_89_p1317}.
Combining this knowledge with \ac{FCS} methods allows one to determine
information about molecular dynamics, concentrations and diffusion
coefficients from images obtained with most \acp{LSCM}. When applied to estimating the \acf{DC} of fluorescently
labelled \ac{ATP} in rat cardiomyocytes \cite{Vendelin_08_AmJPhysiolCellPhysiol_295_pC1302}, a more pronounced reduction was
found compared to that determined by \ac{NMR}. Due to the technical
difficulties experienced in
\cite{Vendelin_08_AmJPhysiolCellPhysiol_295_pC1302} the \ac{RICS} method
was extended further and has developed into what forms the main results
of this dissertation.

The extended \ac{RICS} technique developed herein was applied to
estimate the \acp{DC} of
two different fluorescent molecules (\ATP\ and \DEX ) in rat
cardiomyocytes. The results obtained from these measurements were used
to determine parameters of hypothesized diffusion restrictions in
form of semi-permeable barriers. This was carried out with the aid of a
stochastic computational model of restricted diffusion. The extended
\ac{RICS} method was applied in analysis of both, experimental and
modelled data. \ref{ch:rics} introduces the \ac{RICS} method and the
fundamentals of the extensions to it that have been employed in this
dissertation. It is almost entirely composed of the first chapter of the
Supporting Material of \PaperIII and is included in full to familiarize the
reader with the subject. \ref{ch:exp_res} presents a summary of the
experimental results, while \ref{ch:comp_res} lays out the results obtained
from computational model of restricted diffusion. Information presented
in Chapters 3 to 4 are concise summaries of the
main results given in detail in \PaperIII and its supporting
material. In \ref{ch:1d_comp_res} some hitherto unpublished aspects concerning computational
studies on one-dimensional regional \ac{RICS} are presented.

The second topic explored in this work concerns hypertrophic
cardiomyopathy caused by the Noonan syndrome, which is a relatively common
genetic disorder (affecting 1 in 1000-2500 births
\cite{DelRe_11_JMolCellCardiol_51_p1}). It causes
abnormal development in many parts of the body and can result in
congenital heart defects
\cite{Tartaglia_06_NatGenet_39_p75,P_07_NatGenet_39_p1007}.
\ref{ch:noonan} covers results from the experimental study presented in
full detail in ~\PaperII, where three common mutations responsible for
Noonan syndrome and their effect on downstream signalling pathways are
investigated. Methods employed included immunoblotting for determining changes in expression
levels of several proteins playing vital roles in \ce{Ca^2+} signalling and
confocal fluorescence measurements to establish possible dysfunction of
\ce{Ca^2+} handling.

\ref{ch:mice} summarizes the main conclusions from \PaperI, where
a review of developmental changes in formation of energetic microdomains
in cardiomyocytes was presented.


In summary, this dissertation presents a concise overview of the work
performed during my doctoral studies. With the exception of
\ref{ch:rics}, where the details of the extended \ac{RICS} method have been laid out to
introduce the topic to the reader, chapters present only the main
results and conclusions from each of the studied topics in order to
avoid excessive duplication with respect to the full publications given in the
appendix. The reader is invited to
consult the appended publications to gain more profound insight into
the background information, methods used and physiological implications
of the results obtained.

%\section{List of Publications}
%    %\chapter*{List of Publications}
    \begin{enumerate}
    \item[\PaperNumI ] Illaste A, Kalda M, Schryer DW, Sepp M ; {\bf Life of mice - development of
  cardiac energetics.} \emph{Journal of Physiology}, 588(23), December 2010

    \item[\PaperNumII ] Dhandapany P, Fabris F, Tonk R, Illaste A,
        Karakikes I, Sorourian M, Sheng J, Hajjar R, Tartaglia M, Sobie
        R, Lebeche D,  Gelb B; {\bf Cyclosporine attenuates
        cardiomyocyte hypertrophy induced by RAF1 mutants in Noonan and
        LEOPARD syndromes.} \emph{Journal of Molecular and Cellular Cardiology}, Volume 51, Issue 1, July 2011
    \item[\PaperNumIII ] Illaste A, Laasmaa M, Peterson P, Vendelin M;
        {\bf Analysis of molecular movement reveals latticelike
        obstructions to diffusion in heart mus\-cle cells. } \emph{Biophysical Journal}, 2012, \emph{in press}

    \end{enumerate}

%
\subsection*{Summary of author's contributions}

\begin{enumerate}[I]
  \item Organizer and  main writer for \PaperI. 
  \item For \PaperII, which consists of work carried out in Mount Sinai
    School of Medicine, I carried out some of the confocal microscopy experiments, wrote data analysis software and performed data analysis to ascertain the differences in calcium handling between \acs{RAF1}-mutation-induced hypertrofic and wildtype cardiomyocytes.
  \item In \PaperIII, consisting of the main results of my doctoral studies, I conducted the
\ac{RICS} experiments, improved the design of the experimental protocol, wrote the
code for and performed numerical
simulations together with analysis of the results, and prepared a large part of the
manuscript and figures. 
\end{enumerate}

%
%\section{List of Conference Presentations}
%   % \chapter*{List of Conference Presentations}
    \begin{enumerate}[I]

\item Illaste A, Vendelin M; {\bf Mathematical Model of Mitochondrial
  Energy Metabolism}; 64\textsuperscript{th} Harden Conference on Mitochondrial Physiology, Ambleside, United Kingdom, September 14 - 18, 2007

\item Illaste A, Vendelin M; {\bf Mathematical Model Of Mitochondrial Energy Metabolism
  }; Biophysical Meeting 2008, Long Beach, California, USA, February 2 - 6, 2008

\item Illaste A, Vendelin M; {\bf Computational Model Of Citric Acid
  Cycle And Oxidative Phosphorylation In Mitochondria}; Biophysical
  Meeting 2009, Boston, Massachusetts, February 28 - March 4, 2009

\item Fabris F, Illaste A; Park M; Adler E D; Sobie E A; {\bf
Mechanisms Underlying Spontaneous Beating in Human Embryonic Stem
Cell-Derived Cardiac Myocytes}; Biophysical Meeting 2010, San Francisco,
California, USA, February 20 - 24, 2010

\item Illaste A, Laasmaa M, Schryer D, Birkedal R, Peterson P,
  Vendelin M; {\bf Determination of Regional Diffusion Coefficients
  of Fluorescent ATP in Rat Cardiomyocytes}; Biophysical Meeting 2010, San Francisco,
California, USA, February 20 - 24, 2010
  
\item Illaste A, Laasmaa M, Birkedal R, Peterson P,
  Vendelin M; {\bf Mapping Diffusion Coefficients of Fluorescent
  Dyes in Cardiomyocytes}; Biophysical Meeting 2011, Baltimore, Maryland, USA, March 5 - 9, 2011

\item Illaste A, Laasmaa M, Peterson P, Vendelin M; {\bf
    Analysis of Molecular Movement Reveals Latticelike Obstructions to
    Diffusion in Heart Muscle Cells}; Biophysical Meeting 2012, San Diego, California, USA, February 25 - 29, 2012

    \end{enumerate}

%
%\section{Released software}
%  \textsf{LSJuicer} -- a multi-platform tool for analysing fluorescence vs.
%  time data (\eg, linescan images from confocal microscopes, csv data
%  from spectrophotometers, fluorimeters, etc). This open-source software was used
%  for analyzing \ce{Ca^2+} transients in \PaperII and was also employed for \ce{Ca^2+} spark analysis in
%  \cite{Ramay_11_CardiovascRes_91_p598}. Available at
%  \url{http://lsjuicer.googlecode.com}




